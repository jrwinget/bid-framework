% Options for packages loaded elsewhere
\PassOptionsToPackage{unicode}{hyperref}
\PassOptionsToPackage{hyphens}{url}
\PassOptionsToPackage{dvipsnames,svgnames,x11names}{xcolor}
%
\documentclass[
  authoryear,
  preprint]{elsarticle}

\usepackage{amsmath,amssymb}
\usepackage{iftex}
\ifPDFTeX
  \usepackage[T1]{fontenc}
  \usepackage[utf8]{inputenc}
  \usepackage{textcomp} % provide euro and other symbols
\else % if luatex or xetex
  \usepackage{unicode-math}
  \defaultfontfeatures{Scale=MatchLowercase}
  \defaultfontfeatures[\rmfamily]{Ligatures=TeX,Scale=1}
\fi
\usepackage{lmodern}
\ifPDFTeX\else  
    % xetex/luatex font selection
\fi
% Use upquote if available, for straight quotes in verbatim environments
\IfFileExists{upquote.sty}{\usepackage{upquote}}{}
\IfFileExists{microtype.sty}{% use microtype if available
  \usepackage[]{microtype}
  \UseMicrotypeSet[protrusion]{basicmath} % disable protrusion for tt fonts
}{}
\makeatletter
\@ifundefined{KOMAClassName}{% if non-KOMA class
  \IfFileExists{parskip.sty}{%
    \usepackage{parskip}
  }{% else
    \setlength{\parindent}{0pt}
    \setlength{\parskip}{6pt plus 2pt minus 1pt}}
}{% if KOMA class
  \KOMAoptions{parskip=half}}
\makeatother
\usepackage{xcolor}
\setlength{\emergencystretch}{3em} % prevent overfull lines
\setcounter{secnumdepth}{5}
% Make \paragraph and \subparagraph free-standing
\makeatletter
\ifx\paragraph\undefined\else
  \let\oldparagraph\paragraph
  \renewcommand{\paragraph}{
    \@ifstar
      \xxxParagraphStar
      \xxxParagraphNoStar
  }
  \newcommand{\xxxParagraphStar}[1]{\oldparagraph*{#1}\mbox{}}
  \newcommand{\xxxParagraphNoStar}[1]{\oldparagraph{#1}\mbox{}}
\fi
\ifx\subparagraph\undefined\else
  \let\oldsubparagraph\subparagraph
  \renewcommand{\subparagraph}{
    \@ifstar
      \xxxSubParagraphStar
      \xxxSubParagraphNoStar
  }
  \newcommand{\xxxSubParagraphStar}[1]{\oldsubparagraph*{#1}\mbox{}}
  \newcommand{\xxxSubParagraphNoStar}[1]{\oldsubparagraph{#1}\mbox{}}
\fi
\makeatother


\providecommand{\tightlist}{%
  \setlength{\itemsep}{0pt}\setlength{\parskip}{0pt}}\usepackage{longtable,booktabs,array}
\usepackage{calc} % for calculating minipage widths
% Correct order of tables after \paragraph or \subparagraph
\usepackage{etoolbox}
\makeatletter
\patchcmd\longtable{\par}{\if@noskipsec\mbox{}\fi\par}{}{}
\makeatother
% Allow footnotes in longtable head/foot
\IfFileExists{footnotehyper.sty}{\usepackage{footnotehyper}}{\usepackage{footnote}}
\makesavenoteenv{longtable}
\usepackage{graphicx}
\makeatletter
\newsavebox\pandoc@box
\newcommand*\pandocbounded[1]{% scales image to fit in text height/width
  \sbox\pandoc@box{#1}%
  \Gscale@div\@tempa{\textheight}{\dimexpr\ht\pandoc@box+\dp\pandoc@box\relax}%
  \Gscale@div\@tempb{\linewidth}{\wd\pandoc@box}%
  \ifdim\@tempb\p@<\@tempa\p@\let\@tempa\@tempb\fi% select the smaller of both
  \ifdim\@tempa\p@<\p@\scalebox{\@tempa}{\usebox\pandoc@box}%
  \else\usebox{\pandoc@box}%
  \fi%
}
% Set default figure placement to htbp
\def\fps@figure{htbp}
\makeatother

\makeatletter
\@ifpackageloaded{caption}{}{\usepackage{caption}}
\AtBeginDocument{%
\ifdefined\contentsname
  \renewcommand*\contentsname{Table of contents}
\else
  \newcommand\contentsname{Table of contents}
\fi
\ifdefined\listfigurename
  \renewcommand*\listfigurename{List of Figures}
\else
  \newcommand\listfigurename{List of Figures}
\fi
\ifdefined\listtablename
  \renewcommand*\listtablename{List of Tables}
\else
  \newcommand\listtablename{List of Tables}
\fi
\ifdefined\figurename
  \renewcommand*\figurename{Figure}
\else
  \newcommand\figurename{Figure}
\fi
\ifdefined\tablename
  \renewcommand*\tablename{Table}
\else
  \newcommand\tablename{Table}
\fi
}
\@ifpackageloaded{float}{}{\usepackage{float}}
\floatstyle{ruled}
\@ifundefined{c@chapter}{\newfloat{codelisting}{h}{lop}}{\newfloat{codelisting}{h}{lop}[chapter]}
\floatname{codelisting}{Listing}
\newcommand*\listoflistings{\listof{codelisting}{List of Listings}}
\makeatother
\makeatletter
\makeatother
\makeatletter
\@ifpackageloaded{caption}{}{\usepackage{caption}}
\@ifpackageloaded{subcaption}{}{\usepackage{subcaption}}
\makeatother
\journal{Journal Name}

\usepackage[]{natbib}
\bibliographystyle{elsarticle-harv}
\usepackage{bookmark}

\IfFileExists{xurl.sty}{\usepackage{xurl}}{} % add URL line breaks if available
\urlstyle{same} % disable monospaced font for URLs
\hypersetup{
  pdftitle={Behavioral Insight Design: A Psychological Framework for Designing Data-Driven Decision Support Systems},
  pdfauthor={Jeremy R. Winget; Milena Eickhoff},
  pdfkeywords={keyword1, keyword2},
  colorlinks=true,
  linkcolor={blue},
  filecolor={Maroon},
  citecolor={Blue},
  urlcolor={Blue},
  pdfcreator={LaTeX via pandoc}}


\setlength{\parindent}{6pt}
\begin{document}

\begin{frontmatter}
\title{Behavioral Insight Design: A Psychological Framework for
Designing Data-Driven Decision Support Systems \\\large{Behavioral
Insight Design} }
\author[1]{Jeremy R. Winget%
\corref{cor1}%
\fnref{fn1}}
 \ead{contact@jrwinget.com} 
\author[2]{Milena Eickhoff%
%
\fnref{fn2}}
 \ead{m.eickhoff@skimgroup.com} 

\affiliation[1]{organization={Some Institute of Technology, Department
Name},addressline={Street Address},city={City},postcode={Postal
Code},postcodesep={}}
\affiliation[2]{organization={Another University, Department
Name},addressline={Street Address},city={City},postcode={Postal
Code},postcodesep={}}

\cortext[cor1]{Corresponding author}
\fntext[fn1]{This is the first author footnote.}
\fntext[fn2]{Another author footnote, this is a very long footnote and
it should be a really long footnote. But this footnote is not yet
sufficiently long enough to make two lines of footnote text.}
        
\begin{abstract}
This is the abstract. Lorem ipsum dolor sit amet, consectetur adipiscing
elit. Vestibulum augue turpis, dictum non malesuada a, volutpat eget
velit. Nam placerat turpis purus, eu tristique ex tincidunt et. Mauris
sed augue eget turpis ultrices tincidunt. Sed et mi in leo porta
egestas. Aliquam non laoreet velit. Nunc quis ex vitae eros aliquet
auctor nec ac libero. Duis laoreet sapien eu mi luctus, in bibendum leo
molestie. Sed hendrerit diam diam, ac dapibus nisl volutpat vitae.
Aliquam bibendum varius libero, eu efficitur justo rutrum at. Sed at
tempus elit.
\end{abstract}





\begin{keyword}
    keyword1 \sep 
    keyword2
\end{keyword}
\end{frontmatter}
    

\section{Abstract}\label{abstract}

\begin{itemize}
\tightlist
\item
  Define the Behavioral Insight Design (BID) framework.
\item
  Highlight theoretical foundations (cognitive psychology, behavioral
  economics, dual-process theories).
\item
  Emphasize its operationalization of psychological theory into
  practical design principles.
\item
  State its interdisciplinary value between psychology and UX/HCI.
\end{itemize}

\section{Introduction}\label{introduction}

\begin{itemize}
\tightlist
\item
  Motivation: increased cognitive demands in data-rich environments.
\item
  Identify the gap: lack of structured psychological frameworks in
  UX/HCI.
\item
  Introduce BID as a synthesis of established psychological theories.
\end{itemize}

\section{Theoretical Background}\label{theoretical-background}

Summarize foundational theories relevant to BID: - Cognitive Load Theory
(Sweller, 1988) - Dual-Process Theory (Kahneman \& Tversky) - Cognitive
Biases (Anchoring, Framing, Confirmation Bias) - Information Processing
and Visual Perception (Gestalt principles, Tufte) - Peak--End Rule \&
Aesthetic--Usability Effect (Dion et al., 1972; Norman, 2002)

\section{The Behavioral Insight Design (BID)
Framework}\label{the-behavioral-insight-design-bid-framework}

Outline BID's 5-stage model explicitly mapping psychological theories to
each: - Stage 1: Notice (Cognitive Load, Hick's Law, Visual Hierarchy) -
Stage 2: Interpret (Data Storytelling, Processing Fluency, Emotional
Framing) - Stage 3: Structure (Gestalt Principles, Dual-Processing,
Default Effects) - Stage 4: Anticipate (Anchoring, Framing, Confirmation
Bias, Risk Perception) - Stage 5: Validate \& Empower (Peak--End Rule,
Beautiful-is-Good, Cooperation \& Coordination)

\section{Discussion: Theoretical Implications \&
Innovations}\label{discussion-theoretical-implications-innovations}

\begin{itemize}
\tightlist
\item
  How BID operationalizes psychological concepts into UX steps.
\item
  Novelty in explicitly anticipating and mitigating biases within design
  phases.
\item
  Potential as a tool for cognitive/behavioral psychology
  experimentation.
\end{itemize}

\section{Future Directions}\label{future-directions}

\begin{itemize}
\tightlist
\item
  Propose empirical studies testing BID-derived predictions.
\item
  Suggest deeper investigations into psychological mechanisms
  influencing dashboard efficacy (e.g., cognitive load, memory
  retention, decision accuracy).
\end{itemize}

\section{Conclusion}\label{conclusion}

\begin{itemize}
\tightlist
\item
  Restate BID's theoretical contribution as a coherent synthesis of
  psychology into design methodology.
\end{itemize}


  \bibliography{bibliography.bib}



\end{document}
