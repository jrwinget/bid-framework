% Options for packages loaded elsewhere
\PassOptionsToPackage{unicode}{hyperref}
\PassOptionsToPackage{hyphens}{url}
\PassOptionsToPackage{dvipsnames,svgnames,x11names}{xcolor}
%
\documentclass[
  authoryear,
  preprint]{elsarticle}

\usepackage{amsmath,amssymb}
\usepackage{iftex}
\ifPDFTeX
  \usepackage[T1]{fontenc}
  \usepackage[utf8]{inputenc}
  \usepackage{textcomp} % provide euro and other symbols
\else % if luatex or xetex
  \usepackage{unicode-math}
  \defaultfontfeatures{Scale=MatchLowercase}
  \defaultfontfeatures[\rmfamily]{Ligatures=TeX,Scale=1}
\fi
\usepackage{lmodern}
\ifPDFTeX\else  
    % xetex/luatex font selection
\fi
% Use upquote if available, for straight quotes in verbatim environments
\IfFileExists{upquote.sty}{\usepackage{upquote}}{}
\IfFileExists{microtype.sty}{% use microtype if available
  \usepackage[]{microtype}
  \UseMicrotypeSet[protrusion]{basicmath} % disable protrusion for tt fonts
}{}
\makeatletter
\@ifundefined{KOMAClassName}{% if non-KOMA class
  \IfFileExists{parskip.sty}{%
    \usepackage{parskip}
  }{% else
    \setlength{\parindent}{0pt}
    \setlength{\parskip}{6pt plus 2pt minus 1pt}}
}{% if KOMA class
  \KOMAoptions{parskip=half}}
\makeatother
\usepackage{xcolor}
\setlength{\emergencystretch}{3em} % prevent overfull lines
\setcounter{secnumdepth}{5}
% Make \paragraph and \subparagraph free-standing
\makeatletter
\ifx\paragraph\undefined\else
  \let\oldparagraph\paragraph
  \renewcommand{\paragraph}{
    \@ifstar
      \xxxParagraphStar
      \xxxParagraphNoStar
  }
  \newcommand{\xxxParagraphStar}[1]{\oldparagraph*{#1}\mbox{}}
  \newcommand{\xxxParagraphNoStar}[1]{\oldparagraph{#1}\mbox{}}
\fi
\ifx\subparagraph\undefined\else
  \let\oldsubparagraph\subparagraph
  \renewcommand{\subparagraph}{
    \@ifstar
      \xxxSubParagraphStar
      \xxxSubParagraphNoStar
  }
  \newcommand{\xxxSubParagraphStar}[1]{\oldsubparagraph*{#1}\mbox{}}
  \newcommand{\xxxSubParagraphNoStar}[1]{\oldsubparagraph{#1}\mbox{}}
\fi
\makeatother


\providecommand{\tightlist}{%
  \setlength{\itemsep}{0pt}\setlength{\parskip}{0pt}}\usepackage{longtable,booktabs,array}
\usepackage{calc} % for calculating minipage widths
% Correct order of tables after \paragraph or \subparagraph
\usepackage{etoolbox}
\makeatletter
\patchcmd\longtable{\par}{\if@noskipsec\mbox{}\fi\par}{}{}
\makeatother
% Allow footnotes in longtable head/foot
\IfFileExists{footnotehyper.sty}{\usepackage{footnotehyper}}{\usepackage{footnote}}
\makesavenoteenv{longtable}
\usepackage{graphicx}
\makeatletter
\newsavebox\pandoc@box
\newcommand*\pandocbounded[1]{% scales image to fit in text height/width
  \sbox\pandoc@box{#1}%
  \Gscale@div\@tempa{\textheight}{\dimexpr\ht\pandoc@box+\dp\pandoc@box\relax}%
  \Gscale@div\@tempb{\linewidth}{\wd\pandoc@box}%
  \ifdim\@tempb\p@<\@tempa\p@\let\@tempa\@tempb\fi% select the smaller of both
  \ifdim\@tempa\p@<\p@\scalebox{\@tempa}{\usebox\pandoc@box}%
  \else\usebox{\pandoc@box}%
  \fi%
}
% Set default figure placement to htbp
\def\fps@figure{htbp}
\makeatother

\makeatletter
\@ifpackageloaded{caption}{}{\usepackage{caption}}
\AtBeginDocument{%
\ifdefined\contentsname
  \renewcommand*\contentsname{Table of contents}
\else
  \newcommand\contentsname{Table of contents}
\fi
\ifdefined\listfigurename
  \renewcommand*\listfigurename{List of Figures}
\else
  \newcommand\listfigurename{List of Figures}
\fi
\ifdefined\listtablename
  \renewcommand*\listtablename{List of Tables}
\else
  \newcommand\listtablename{List of Tables}
\fi
\ifdefined\figurename
  \renewcommand*\figurename{Figure}
\else
  \newcommand\figurename{Figure}
\fi
\ifdefined\tablename
  \renewcommand*\tablename{Table}
\else
  \newcommand\tablename{Table}
\fi
}
\@ifpackageloaded{float}{}{\usepackage{float}}
\floatstyle{ruled}
\@ifundefined{c@chapter}{\newfloat{codelisting}{h}{lop}}{\newfloat{codelisting}{h}{lop}[chapter]}
\floatname{codelisting}{Listing}
\newcommand*\listoflistings{\listof{codelisting}{List of Listings}}
\makeatother
\makeatletter
\makeatother
\makeatletter
\@ifpackageloaded{caption}{}{\usepackage{caption}}
\@ifpackageloaded{subcaption}{}{\usepackage{subcaption}}
\makeatother
\journal{Journal Name}

\usepackage[]{natbib}
\bibliographystyle{elsarticle-harv}
\usepackage{bookmark}

\IfFileExists{xurl.sty}{\usepackage{xurl}}{} % add URL line breaks if available
\urlstyle{same} % disable monospaced font for URLs
\hypersetup{
  pdftitle={Applying Behavioral Insight Design: A Practical Framework for Enhancing User Experience in Interactive Data Applications},
  pdfauthor={Milena Eickhoff; Jeremy R. Winget},
  pdfkeywords={keyword1, keyword2},
  colorlinks=true,
  linkcolor={blue},
  filecolor={Maroon},
  citecolor={Blue},
  urlcolor={Blue},
  pdfcreator={LaTeX via pandoc}}


\setlength{\parindent}{6pt}
\begin{document}

\begin{frontmatter}
\title{Applying Behavioral Insight Design: A Practical Framework for
Enhancing User Experience in Interactive Data
Applications \\\large{Applying BID Framework} }
\author[1]{Milena Eickhoff%
\corref{cor1}%
\fnref{fn1}}
 \ead{m.eickhoff@skimgroup.com} 
\author[2]{Jeremy R. Winget%
%
\fnref{fn2}}
 \ead{contact@jrwinget.com} 

\affiliation[1]{organization={Another University, Department
Name},addressline={Street Address},city={City},postcode={Postal
Code},postcodesep={}}
\affiliation[2]{organization={Some Institute of Technology, Department
Name},addressline={Street Address},city={City},postcode={Postal
Code},postcodesep={}}

\cortext[cor1]{Corresponding author}
\fntext[fn1]{This is the first author footnote.}
\fntext[fn2]{Another author footnote, this is a very long footnote and
it should be a really long footnote. But this footnote is not yet
sufficiently long enough to make two lines of footnote text.}
        
\begin{abstract}
This is the abstract. Lorem ipsum dolor sit amet, consectetur adipiscing
elit. Vestibulum augue turpis, dictum non malesuada a, volutpat eget
velit. Nam placerat turpis purus, eu tristique ex tincidunt et. Mauris
sed augue eget turpis ultrices tincidunt. Sed et mi in leo porta
egestas. Aliquam non laoreet velit. Nunc quis ex vitae eros aliquet
auctor nec ac libero. Duis laoreet sapien eu mi luctus, in bibendum leo
molestie. Sed hendrerit diam diam, ac dapibus nisl volutpat vitae.
Aliquam bibendum varius libero, eu efficitur justo rutrum at. Sed at
tempus elit.
\end{abstract}





\begin{keyword}
    keyword1 \sep 
    keyword2
\end{keyword}
\end{frontmatter}
    

\section{Abstract}\label{abstract}

\begin{itemize}
\tightlist
\item
  Briefly introduce the BID framework.
\item
  Emphasize practical applications: structured steps
  designers/developers can directly adopt.
\item
  Highlight UX advantages: reduced cognitive friction, enhanced insight
  extraction, increased user satisfaction.
\end{itemize}

\section{Introduction}\label{introduction}

\begin{itemize}
\tightlist
\item
  Context: growing industry demand for actionable UX methodologies
  incorporating psychological insights.
\item
  State the need for a clear, structured UX process grounded in
  cognitive science and behavioral insights.
\item
  Introduce BID as a practical UX-oriented framework designed explicitly
  for interactive data applications.
\end{itemize}

\section{Literature Review: Practical UX
Context}\label{literature-review-practical-ux-context}

\begin{itemize}
\tightlist
\item
  Briefly overview existing UX frameworks (Design with Intent,
  Persuasive Systems Design, Cognitive Fit Theory, data storytelling).
\item
  Identify practical limitations or gaps in existing methods (e.g.,
  fragmented approaches, lack of a structured end-to-end method).
\item
  Position BID clearly within the UX literature as an integrated
  solution, drawing explicitly on the psychological theories outlined in
  the theoretical paper.
\end{itemize}

\section{Overview of Behavioral Insight Design (BID)
Framework}\label{overview-of-behavioral-insight-design-bid-framework}

\begin{itemize}
\tightlist
\item
  Short summary of each stage tailored specifically for UX
  practitioners:

  \begin{itemize}
  \tightlist
  \item
    Notice user friction
  \item
    Interpret user insights
  \item
    Structure dashboard information
  \item
    Anticipate cognitive biases
  \item
    Validate and empower users through UX design
  \end{itemize}
\end{itemize}

\section{Application of the BID Framework in
Practice}\label{application-of-the-bid-framework-in-practice}

For each stage, include: 1. Stage 1: Notice User Friction - Practical
methods: heuristic evaluations, friction point audits, user interviews.
- UX tools recommended: usability testing, cognitive walkthroughs. 1.
Stage 2: Interpret User Insights - Methods: data storytelling best
practices, storyboard creation, defining user journeys. - Highlight
practical tips for narrative clarity and data reduction. 1. Stage 3:
Structure Dashboard Information - Guidelines: visual hierarchy,
defaults, Gestalt grouping. - Practical UX examples: optimal layouts,
recommended widgets/components, interactivity levels. 1. Stage 4:
Anticipate Cognitive Biases - Explicit bias mitigation techniques:
visual framing, scenario toggles, debiasing cues. - Provide examples of
bias-aware visual designs (e.g., anchoring reduction through careful
baseline selection, reframing of KPIs). 1. Stage 5: Validate and Empower
Users - Recommendations: summarize key insights, ensure aesthetic
appeal, implement collaborative UX features. - UX best practices: design
checklists for peak-end experiences, validation via usability testing.

\section{Case Study: Implementation of BID
Framework}\label{case-study-implementation-of-bid-framework}

\begin{itemize}
\tightlist
\item
  Showcase a detailed, practical example (ideally using a real or
  hypothetical Shiny dashboard case).
\item
  Highlight how BID practically influenced UX design choices at each
  stage.
\item
  Discuss measurable UX outcomes: improved usability, increased user
  comprehension/confidence, evidence from user testing, or satisfaction
  scores.
\end{itemize}

\section{Discussion: UX Impact \& Practical
Adoption}\label{discussion-ux-impact-practical-adoption}

\begin{itemize}
\tightlist
\item
  Summarize how BID fills UX gaps and provides actionable guidance.
\item
  Highlight benefits specifically for dashboard or interactive data
  visualization designers.
\item
  Discuss practitioner-oriented implications: ease of adoption, required
  UX skill levels, integration into agile UX workflows.
\end{itemize}

\section{Future UX Research and
Extensions}\label{future-ux-research-and-extensions}

\begin{itemize}
\tightlist
\item
  Suggest practical UX studies to empirically validate BID framework
  effectiveness.
\item
  Outline potential UX-focused extensions: accessibility design,
  inclusive UX practices, AI-assisted BID integration.
\end{itemize}

\section{Conclusion}\label{conclusion}

\begin{itemize}
\tightlist
\item
  Restate BID's value for UX practice: clear, structured, actionable.
\item
  Encourage UX professionals to adopt and adapt the BID framework in
  industry settings.
\end{itemize}


  \bibliography{bibliography.bib}



\end{document}
