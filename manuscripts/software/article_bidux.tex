% Options for packages loaded elsewhere
% Options for packages loaded elsewhere
\PassOptionsToPackage{unicode}{hyperref}
\PassOptionsToPackage{hyphens}{url}
\PassOptionsToPackage{dvipsnames,svgnames,x11names}{xcolor}
%
\documentclass[
  authoryear,
  preprint]{elsarticle}
\usepackage{xcolor}
\usepackage{amsmath,amssymb}
\setcounter{secnumdepth}{5}
\usepackage{iftex}
\ifPDFTeX
  \usepackage[T1]{fontenc}
  \usepackage[utf8]{inputenc}
  \usepackage{textcomp} % provide euro and other symbols
\else % if luatex or xetex
  \usepackage{unicode-math} % this also loads fontspec
  \defaultfontfeatures{Scale=MatchLowercase}
  \defaultfontfeatures[\rmfamily]{Ligatures=TeX,Scale=1}
\fi
\usepackage{lmodern}
\ifPDFTeX\else
  % xetex/luatex font selection
\fi
% Use upquote if available, for straight quotes in verbatim environments
\IfFileExists{upquote.sty}{\usepackage{upquote}}{}
\IfFileExists{microtype.sty}{% use microtype if available
  \usepackage[]{microtype}
  \UseMicrotypeSet[protrusion]{basicmath} % disable protrusion for tt fonts
}{}
\makeatletter
\@ifundefined{KOMAClassName}{% if non-KOMA class
  \IfFileExists{parskip.sty}{%
    \usepackage{parskip}
  }{% else
    \setlength{\parindent}{0pt}
    \setlength{\parskip}{6pt plus 2pt minus 1pt}}
}{% if KOMA class
  \KOMAoptions{parskip=half}}
\makeatother
% Make \paragraph and \subparagraph free-standing
\makeatletter
\ifx\paragraph\undefined\else
  \let\oldparagraph\paragraph
  \renewcommand{\paragraph}{
    \@ifstar
      \xxxParagraphStar
      \xxxParagraphNoStar
  }
  \newcommand{\xxxParagraphStar}[1]{\oldparagraph*{#1}\mbox{}}
  \newcommand{\xxxParagraphNoStar}[1]{\oldparagraph{#1}\mbox{}}
\fi
\ifx\subparagraph\undefined\else
  \let\oldsubparagraph\subparagraph
  \renewcommand{\subparagraph}{
    \@ifstar
      \xxxSubParagraphStar
      \xxxSubParagraphNoStar
  }
  \newcommand{\xxxSubParagraphStar}[1]{\oldsubparagraph*{#1}\mbox{}}
  \newcommand{\xxxSubParagraphNoStar}[1]{\oldsubparagraph{#1}\mbox{}}
\fi
\makeatother

\usepackage{color}
\usepackage{fancyvrb}
\newcommand{\VerbBar}{|}
\newcommand{\VERB}{\Verb[commandchars=\\\{\}]}
\DefineVerbatimEnvironment{Highlighting}{Verbatim}{commandchars=\\\{\}}
% Add ',fontsize=\small' for more characters per line
\usepackage{framed}
\definecolor{shadecolor}{RGB}{241,243,245}
\newenvironment{Shaded}{\begin{snugshade}}{\end{snugshade}}
\newcommand{\AlertTok}[1]{\textcolor[rgb]{0.68,0.00,0.00}{#1}}
\newcommand{\AnnotationTok}[1]{\textcolor[rgb]{0.37,0.37,0.37}{#1}}
\newcommand{\AttributeTok}[1]{\textcolor[rgb]{0.40,0.45,0.13}{#1}}
\newcommand{\BaseNTok}[1]{\textcolor[rgb]{0.68,0.00,0.00}{#1}}
\newcommand{\BuiltInTok}[1]{\textcolor[rgb]{0.00,0.23,0.31}{#1}}
\newcommand{\CharTok}[1]{\textcolor[rgb]{0.13,0.47,0.30}{#1}}
\newcommand{\CommentTok}[1]{\textcolor[rgb]{0.37,0.37,0.37}{#1}}
\newcommand{\CommentVarTok}[1]{\textcolor[rgb]{0.37,0.37,0.37}{\textit{#1}}}
\newcommand{\ConstantTok}[1]{\textcolor[rgb]{0.56,0.35,0.01}{#1}}
\newcommand{\ControlFlowTok}[1]{\textcolor[rgb]{0.00,0.23,0.31}{\textbf{#1}}}
\newcommand{\DataTypeTok}[1]{\textcolor[rgb]{0.68,0.00,0.00}{#1}}
\newcommand{\DecValTok}[1]{\textcolor[rgb]{0.68,0.00,0.00}{#1}}
\newcommand{\DocumentationTok}[1]{\textcolor[rgb]{0.37,0.37,0.37}{\textit{#1}}}
\newcommand{\ErrorTok}[1]{\textcolor[rgb]{0.68,0.00,0.00}{#1}}
\newcommand{\ExtensionTok}[1]{\textcolor[rgb]{0.00,0.23,0.31}{#1}}
\newcommand{\FloatTok}[1]{\textcolor[rgb]{0.68,0.00,0.00}{#1}}
\newcommand{\FunctionTok}[1]{\textcolor[rgb]{0.28,0.35,0.67}{#1}}
\newcommand{\ImportTok}[1]{\textcolor[rgb]{0.00,0.46,0.62}{#1}}
\newcommand{\InformationTok}[1]{\textcolor[rgb]{0.37,0.37,0.37}{#1}}
\newcommand{\KeywordTok}[1]{\textcolor[rgb]{0.00,0.23,0.31}{\textbf{#1}}}
\newcommand{\NormalTok}[1]{\textcolor[rgb]{0.00,0.23,0.31}{#1}}
\newcommand{\OperatorTok}[1]{\textcolor[rgb]{0.37,0.37,0.37}{#1}}
\newcommand{\OtherTok}[1]{\textcolor[rgb]{0.00,0.23,0.31}{#1}}
\newcommand{\PreprocessorTok}[1]{\textcolor[rgb]{0.68,0.00,0.00}{#1}}
\newcommand{\RegionMarkerTok}[1]{\textcolor[rgb]{0.00,0.23,0.31}{#1}}
\newcommand{\SpecialCharTok}[1]{\textcolor[rgb]{0.37,0.37,0.37}{#1}}
\newcommand{\SpecialStringTok}[1]{\textcolor[rgb]{0.13,0.47,0.30}{#1}}
\newcommand{\StringTok}[1]{\textcolor[rgb]{0.13,0.47,0.30}{#1}}
\newcommand{\VariableTok}[1]{\textcolor[rgb]{0.07,0.07,0.07}{#1}}
\newcommand{\VerbatimStringTok}[1]{\textcolor[rgb]{0.13,0.47,0.30}{#1}}
\newcommand{\WarningTok}[1]{\textcolor[rgb]{0.37,0.37,0.37}{\textit{#1}}}

\usepackage{longtable,booktabs,array}
\usepackage{calc} % for calculating minipage widths
% Correct order of tables after \paragraph or \subparagraph
\usepackage{etoolbox}
\makeatletter
\patchcmd\longtable{\par}{\if@noskipsec\mbox{}\fi\par}{}{}
\makeatother
% Allow footnotes in longtable head/foot
\IfFileExists{footnotehyper.sty}{\usepackage{footnotehyper}}{\usepackage{footnote}}
\makesavenoteenv{longtable}
\usepackage{graphicx}
\makeatletter
\newsavebox\pandoc@box
\newcommand*\pandocbounded[1]{% scales image to fit in text height/width
  \sbox\pandoc@box{#1}%
  \Gscale@div\@tempa{\textheight}{\dimexpr\ht\pandoc@box+\dp\pandoc@box\relax}%
  \Gscale@div\@tempb{\linewidth}{\wd\pandoc@box}%
  \ifdim\@tempb\p@<\@tempa\p@\let\@tempa\@tempb\fi% select the smaller of both
  \ifdim\@tempa\p@<\p@\scalebox{\@tempa}{\usebox\pandoc@box}%
  \else\usebox{\pandoc@box}%
  \fi%
}
% Set default figure placement to htbp
\def\fps@figure{htbp}
\makeatother





\setlength{\emergencystretch}{3em} % prevent overfull lines

\providecommand{\tightlist}{%
  \setlength{\itemsep}{0pt}\setlength{\parskip}{0pt}}



 
\usepackage[]{natbib}
\bibliographystyle{elsarticle-harv}


\makeatletter
\@ifpackageloaded{caption}{}{\usepackage{caption}}
\AtBeginDocument{%
\ifdefined\contentsname
  \renewcommand*\contentsname{Table of contents}
\else
  \newcommand\contentsname{Table of contents}
\fi
\ifdefined\listfigurename
  \renewcommand*\listfigurename{List of Figures}
\else
  \newcommand\listfigurename{List of Figures}
\fi
\ifdefined\listtablename
  \renewcommand*\listtablename{List of Tables}
\else
  \newcommand\listtablename{List of Tables}
\fi
\ifdefined\figurename
  \renewcommand*\figurename{Figure}
\else
  \newcommand\figurename{Figure}
\fi
\ifdefined\tablename
  \renewcommand*\tablename{Table}
\else
  \newcommand\tablename{Table}
\fi
}
\@ifpackageloaded{float}{}{\usepackage{float}}
\floatstyle{ruled}
\@ifundefined{c@chapter}{\newfloat{codelisting}{h}{lop}}{\newfloat{codelisting}{h}{lop}[chapter]}
\floatname{codelisting}{Listing}
\newcommand*\listoflistings{\listof{codelisting}{List of Listings}}
\makeatother
\makeatletter
\makeatother
\makeatletter
\@ifpackageloaded{caption}{}{\usepackage{caption}}
\@ifpackageloaded{subcaption}{}{\usepackage{subcaption}}
\makeatother
\journal{xxx}
\usepackage{bookmark}
\IfFileExists{xurl.sty}{\usepackage{xurl}}{} % add URL line breaks if available
\urlstyle{same}
\hypersetup{
  pdftitle={bidux: Behavioral Insight Design for Shiny Dashboards},
  pdfauthor={Jeremy R. Winget, PhD},
  pdfkeywords={R, shiny, behavioral science, UX, human-computer
interaction},
  colorlinks=true,
  linkcolor={blue},
  filecolor={Maroon},
  citecolor={Blue},
  urlcolor={Blue},
  pdfcreator={LaTeX via pandoc}}


\setlength{\parindent}{6pt}
\begin{document}

\begin{frontmatter}
\title{bidux: Behavioral Insight Design for Shiny Dashboards}
\author[1]{Jeremy R. Winget, PhD%
\corref{cor1}%
}
 \ead{contact@jrwinget.com} 

\affiliation[1]{organization={Independent
Researcher},city={Chicago},postcode={60660},postcodesep={}}

\cortext[cor1]{Corresponding author}

        
\begin{abstract}
Dashboards often degrade into sprawling filter farms: impressive on
paper, but opaque to users, brittle in use, and shaped by improvisation
rather than evidence. \texttt{bidux} is an R package that brings a
behavioral-science workflow to Shiny development. It formalizes five
stages (Interpret, Notice, Anticipate, Structure, Validate) and
optionally links them to telemetry so that UI decisions are driven by
observed cognitive friction rather than guesswork. This article
describes the design motivations, theoretical grounding, architecture,
and usage sketch of \texttt{bidux}, and situates it within the
UX/HCI/decision-making literature.
\end{abstract}





\begin{keyword}
    R \sep shiny \sep behavioral science \sep UX \sep 
    human-computer interaction
\end{keyword}
\end{frontmatter}
    

\section{Summary}\label{summary}

Dashboards rarely lack features. What they often lack is \emph{insight},
a bridge between domain logic and how real humans use the interface. I
created \texttt{bidux} to plug that gap. It embeds a behavioral-science
workflow into existing Shiny applications, helping teams move from
gut-driven layout decisions to structured, auditable, cognitively
grounded UI changes.

At its simplest, \texttt{bidux} takes a compact set of design inputs
(e.g.~a central question, user personas, friction notes) and produces a
recommended roadmap: which UI levers to tune (e.g.~reduce filter count,
adopt progressive disclosure, reorder content zones) and why. If you
have telemetry, \texttt{bidux} can highlight the biggest bottlenecks in
the user journey, linking interventions to measurable gains.

\texttt{bidux} is designed for R/Shiny practitioners who care about
usability, but can't always hire a UX specialist. It augments, rather
than replaces, existing Shiny best practices with a disciplined,
theory-informed workflow.

\section{Statement of Need}\label{statement-of-need}

In my consulting and dashboard audits, I repeatedly encountered
dashboards where drop-downs proliferated, click counts ballooned, and
user engagement dropped fast. Many dashboards are constructed by
functional experimenters, domain experts, or statisticians, rarely by
designers. When heuristics or UX reviews arise, they tend to be applied
ad hoc, not baked into the development process.

Researchers and practitioners alike need a middle path: a lightweight,
repeatable, theory-informed workflow for designing dashboards that
respect real human cognitive constraints (choice overload, poor
scanning, information scent failure). \texttt{bidux} addresses precisely
that gap. It helps democratize behavioral design in dashboards by making
the core decisions auditable, revisitable, and repeatable.

Moreover, in research contexts, dashboards often come with a need for
interpretability and justification: stakeholders want to see \emph{why}
a UI change was made, not just that it improved engagement.
\texttt{bidux} provides that rationale trail.

\section{Theoretical \& Empirical
Foundations}\label{theoretical-empirical-foundations}

I won't review every HCI or design book here. Instead, I anchor two
theoretical pillars that \texttt{bidux} leans on heavily:

\subsection{Choice overload \& decision
friction}\label{choice-overload-decision-friction}

The paradox that more options can \emph{hurt} decision making is well
documented. A meta-analysis of 99 experimental conditions (N = 7,202)
identified that large option sets systematically degrade satisfaction,
increase regret, and more frequently lead to choice deferral,
particularly when decision tasks are complex or user preferences
uncertain (i.e.~moderate or ambivalent) \citep{chernev2015}. The effect
is moderated by task difficulty, preference certainty, and cognitive
load.

More recent work argues that many traditional tests of choice overload
are underpowered and that richer data often confirm that the phenomenon
is more common than earlier studies suggested \citep{dean2022}.

This supports a core design rule: fewer competing controls or clearer
defaults often reduce user paralysis and friction.

\subsection{Visual hierarchy \& scanning
behavior}\label{visual-hierarchy-scanning-behavior}

Empirical eye-tracking studies have repeatedly shown that users scan
screens in an ``F-pattern'', starting top-left, moving horizontally,
then vertically and again horizontally, especially in Western reading
contexts \citep{djamasbi2011, nngroup2017}. Information buried low or
far right is often missed entirely.

Further, layout complexity increases scattering of attention; simpler,
well-ordered layouts help preserve the scanning trajectory
\citep{djamasbi2011}. More recent GUI studies reaffirm that users'
visual search in graphical interfaces still reflect reading-based
scanning heuristics \citep{putkonen2025}.

Thus, interface structure matters, not just number of options, but where
and how they appear relative to the scanning eye.

\section{\texorpdfstring{Design \& Architecture of
\texttt{bidux}}{Design \& Architecture of bidux}}\label{design-architecture-of-bidux}

Below I describe the mental model (workflow), key modules, and API
sketch (high level).

\subsection{Behavioral design workflow (five
stages)}\label{behavioral-design-workflow-five-stages}

\begin{enumerate}
\def\labelenumi{\arabic{enumi}.}
\item
  \textbf{Interpret} Define the \emph{central question} your dashboard
  should support, describe context (tension, trade-offs), articulate 1-3
  personas (users and their motivations).
\item
  \textbf{Notice} Record concrete design problems or usage observations
  (e.g.~``users ignore side filters'', ``clicks drop off after two
  filters''). Optionally map each to behavioral theory (e.g.~``choice
  overload'', ``search cost'').
\item
  \textbf{Anticipate} For each design problem, sketch likely cognitive
  failures (e.g.~overload, choice deferral, ordering bias) and propose
  candidate mitigations (defaults, guardrails, pre-filtering, grouping).
\item
  \textbf{Structure} Lay out a candidate UI hierarchy (e.g.~``first
  meaningful view,'' high-salience zones, prioritized filters).
  Internally, \texttt{bidux} leverages a lookup table that maps
  behavioral risk ↦ recommended Shiny interventions
  (accordion/progressive disclosure, reorder panels, sticky headers,
  minimal defaults, grouped controls, etc).
\item
  \textbf{Validate} Check consistency, perform lightweight heuristics
  (e.g.~``does the first view align with the central question?''). If
  telemetry is available, identify the largest drop-offs in the user
  funnel and re-prioritize design fixes. Export a summary ``rationale +
  intervention'' report for stakeholders.
\end{enumerate}

\subsection{Key properties \& architectural
choices}\label{key-properties-architectural-choices}

\begin{itemize}
\tightlist
\item
  \textbf{Telemetry-optional}: \texttt{bidux} functions well without
  usage logs; but if metrics like page views, click events, filter
  application rates are present, \texttt{bidux} ingests them to
  spotlight bottlenecks.
\item
  \textbf{Auditable outputs}: every design decision is stored in tidy
  data frames so you can trace ``Why did I reorder filters?'' back to a
  behavioral rationale.
\item
  \textbf{Override-friendly}: users may override the default mapping of
  behavioral issues → UI interventions; \texttt{bidux} emphasizes
  transparency rather than blind automation.
\item
  \textbf{Composable}: \texttt{bidux} is meant to be used alongside
  existing UI toolkits (e.g.~\texttt{bslib}, \texttt{shiny.semantic})
  and testing frameworks, not replace them.
\end{itemize}

\section{Example Use Sketch}\label{example-use-sketch}

Here's a minimal R pipeline illustrating how \texttt{bidux} might be
used in practice. (Note: this is illustrative; see package docs for full
usage.)

\begin{Shaded}
\begin{Highlighting}[]
\FunctionTok{library}\NormalTok{(bidux)}

\NormalTok{plan }\OtherTok{\textless{}{-}} \FunctionTok{bid\_interpret}\NormalTok{(}
  \AttributeTok{central\_question =} \StringTok{"How can we improve awareness?"}\NormalTok{,}
  \AttributeTok{data\_story =} \FunctionTok{list}\NormalTok{(}
    \AttributeTok{hook =} \StringTok{"Users are missing key messaging"}\NormalTok{,}
    \AttributeTok{context =} \StringTok{"Ad campaigns are running but engagement is low"}\NormalTok{,}
    \AttributeTok{tension =} \StringTok{"Too many competing visuals"}\NormalTok{,}
    \AttributeTok{resolution =} \StringTok{"Simplify and prioritize key elements"}
\NormalTok{  ),}
  \AttributeTok{user\_personas =} \FunctionTok{list}\NormalTok{(}
    \FunctionTok{list}\NormalTok{(}
      \AttributeTok{name =} \StringTok{"Data Analyst"}\NormalTok{,}
      \AttributeTok{goals =} \StringTok{"Needs to quickly find patterns in data"}\NormalTok{,}
      \AttributeTok{pain\_points =} \StringTok{"Gets overwhelmed by too many tables"}\NormalTok{,}
      \AttributeTok{technical\_level =} \StringTok{"Advanced"}
\NormalTok{    ),}
    \FunctionTok{list}\NormalTok{(}
      \AttributeTok{name =} \StringTok{"Executive"}\NormalTok{,}
      \AttributeTok{goals =} \StringTok{"Wants high{-}level insights at a glance"}\NormalTok{,}
      \AttributeTok{pain\_points =} \StringTok{"Limited time to analyze detailed reports"}\NormalTok{,}
      \AttributeTok{technical\_level =} \StringTok{"Basic"}
\NormalTok{    )}
\NormalTok{  )}
\NormalTok{) }\SpecialCharTok{|\textgreater{}}
  \FunctionTok{bid\_notice}\NormalTok{(}
    \AttributeTok{problem =} \StringTok{"Too many filters"}\NormalTok{,}
    \AttributeTok{evidence =} \StringTok{"Users never applied more than 2"}
\NormalTok{  ) }\SpecialCharTok{|\textgreater{}}
  \FunctionTok{bid\_anticipate}\NormalTok{() }\SpecialCharTok{|\textgreater{}}
  \FunctionTok{bid\_structure}\NormalTok{() }\SpecialCharTok{|\textgreater{}}
  \FunctionTok{bid\_validate}\NormalTok{()}

\FunctionTok{print}\NormalTok{(plan}\SpecialCharTok{$}\NormalTok{rationale)}
\FunctionTok{print}\NormalTok{(plan}\SpecialCharTok{$}\NormalTok{interventions)}
\end{Highlighting}
\end{Shaded}

If telemetry data exists, you can inject it before
\texttt{bid\_structure()} to highlight drop-points in the application
funnel and adjust priorities accordingly.

\section{Illustrative Application}\label{illustrative-application}

In a dashboard I built for an (undisclosed) internal research project,
\texttt{bidux} guided me to (a) collapse 12 filter inputs into 4 default
yes/no toggles, (b) reorder the primary outcome display to appear above
controls, and (c) adopt accordion panels for secondary filters. After
deployment, heatmap and click metrics (outside the purview of this
paper) suggested that \textgreater50\% more users applied a first filter
and revisit rates rose modestly. The point is not the magnitude, but
consistency of direction: you get \emph{design with justification}, not
magic.

\section{Availability \& Maturity}\label{availability-maturity}

\texttt{bidux} is available under an OSI-approved MIT license on CRAN
and GitHub. The repository includes function documentation, vignettes,
unit tests, and contribution guidelines. Upon acceptance I will issue a
curated release and archive via Zenodo, minting a software DOI.

\section{Conclusion \& Future
Directions}\label{conclusion-future-directions}

\texttt{bidux} is a tool for the mid-ground: between naïve defaulted UI
wrappers and full UX redesign. It embeds behavioral reasoning into the
Shiny workflow so that layout choices, control counts, and content
ordering aren't whimsical; they're auditable and revisitable.

Future work includes:

\begin{itemize}
\tightlist
\item
  Richer telemetry modules (e.g.~session path mining, A/B test support)
\item
  Integration with browser instrumentation (JavaScript hooks)
\item
  Empirical validation (e.g.~randomized UI experiments using
  \texttt{bidux} vs baseline dashboards)
\item
  Support for non-Shiny front ends or hybrid R + JS dashboards
\end{itemize}

I hope \texttt{bidux} lowers the bar for better dashboards in research
software and helps teams build interfaces that respect human constraints
without discarding their domain logic.

\section{Acknowledgements}\label{acknowledgements}

Thanks to the Posit and Shiny community for feedback on early
prototypes, and to early users who shared telemetry and use cases (with
permission). I also acknowledge conversations with UX colleagues who
patiently listened to dashboard horror stories.


\bibliography{references.bib}



\end{document}
